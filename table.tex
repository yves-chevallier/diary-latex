\documentclass{article}
\usepackage[french]{babel}
\usepackage{tabularx}
\usepackage{lipsum}
\usepackage{booktabs}
\usepackage{listings}
\usepackage{hyperref}
\usepackage{multirow}
\usepackage[table,xcdraw]{xcolor}
\lstset{
    language=C++,
    basicstyle=\tt,
    frame=tb,
    keywordstyle=\color{blue}\bfseries,
    }


\hypersetup{
        colorlinks,
        linkcolor={red!50!black},
        citecolor={blue!50!black},
        urlcolor={blue!80!black}
}

\begin{document}

\listoffigures
\listoftables
\lstlistoflistings

\section{Extrait de code}

\begin{lstlisting}[caption={My Caption},captionpos=b]
#include <stdio.h>

int main(void)
{
    for (int i = 0; i < 10; i++)
        printf("Hello, world!");
    return 42;
}
\end{lstlisting}

\section{Tableaux}

\begin{table}[h]
\centering
    \caption{Tableau 1 \label{tb:tableau1}}
    \begin{tabular}{|lp{5cm}|r||}\hline
        Id & Name & Age \\
        \hline\hline
        1 & John & 25 \\ \hline
        2 & Mary & 23 \\ 
        3 & Peter & 27 \\ \hline
    \end{tabular}
\end{table}    

\begin{table}[h]
    \centering
        \caption{Tableau 1 \label{tb:tableau1}}
        \begin{tabularx}{\textwidth}{lXr}
            \toprule
            Id & Name & Age \\
            \midrule
            1 & John & 25 \\ 
            2 & Mary & 23 \\ 
            3 & Peter & 27 \\ 
            \bottomrule
        \end{tabularx}
    \end{table}    
\lipsum[1-2]

\section{Tableaux avancés}


\begin{table}[h]
    \centering
    \caption{Tableau avancé}
    \begin{tabular}{lclll}
    \hline
    \multicolumn{1}{|l|}{1} & \multicolumn{1}{c|}{2} & \multicolumn{1}{l|}{\textbf{3}} & \multicolumn{1}{l|}{4} & \multicolumn{1}{l|}{5}     \\ \hline
                            & 7                      & \textbf{8}                      & 9                      & 10                         \\
    \multirow{-2}{*}{6}     & 12                     & \textbf{13}                     & 14                     & \cellcolor[HTML]{FE0000}15 \\
    \multicolumn{2}{l}{16}                           & \textbf{18}                     & 19                     & 20                        
    \end{tabular}
    \end{table}
\end{document}
